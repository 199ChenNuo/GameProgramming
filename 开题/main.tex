\documentclass{ctexart}
% Other classes are available too:
% \documentclass{ctexrep}
% \documentclass{ctexbook}
% \documentclass{ctexbeamer}

%% You can change the font if necessary.
% \setCJKmainfont{BabelStone Han}
\setCJKsansfont{Noto Sans CJK SC}
\usepackage{geometry}   %设置页边距的宏包
\usepackage{titlesec}   %设置页眉页脚的宏包

\geometry{left=3cm,right=2.5cm,top=2.5cm,bottom=2.5cm} 

\title{折纸仿真引擎\\\ \\ 开题报告}
\author{
罗宇辰 516030910101  \\\
陈志扬 516030910347  \\\
陈  诺 516030910199
}
\date{2019年5月4日}

\clearpage
\newpage

\begin{document}
\maketitle
\tableofcontents


\begin{abstract}
折纸不仅仅是一种有趣的游戏,其中也蕴含着许多知识,如几何、拓扑变换、微积分等。我们想要设计实现一个折纸仿真引擎,对纸张进行建模,并实现 VR 环境下的折纸模拟与交互。
\end{abstract}

\section{要实现的功能及预期效果}


\section{关于数学部分}
参考论文\cite{ref1}中的公式:
\begin{equation}
    \vec F_{crease} = - k_{crease} (\theta - \theta_{target}) \frac {\partial \theta}{\partial \vec p}
\end{equation}

\begin{equation}
\vec F_{crease} = - k_{crease} (\theta - \theta_{target}) \frac {\partial \theta}{\partial \vec p}
\end{equation}

\begin{equation}
\frac{\partial \theta}{\partial \vec p_1} = \frac{\vec n_1}{h_1}
\end{equation}

\begin{equation}
\frac{\partial \theta}{\partial \vec p_2} = \frac{\vec n_2}{h_2}
\end{equation}

\begin{equation}
\frac{\partial \theta}{\partial \vec p_3} = \frac{-cot \alpha_{4,31}}{cot \alpha_{3,14}+cot \alpha_{4,31}}\frac{\vec n_1}{h_1} + \frac{-cot \alpha_{4,23}}{cot \alpha_{3,42}+cot \alpha_{4,23}}\frac{\vec n_2}{h_2}
\end{equation}

\begin{equation}
\frac{\partial \theta}{\partial \vec p_4} = \frac{-cot \alpha_{3,14}}{cot \alpha_{3,14}+cot \alpha_{4,31}}\frac{\vec n_1}{h_1} + \frac{-cot \alpha_{3,42}}{cot \alpha_{3,42}+cot \alpha_{4,23}}\frac{\vec n_2}{h_2}
\end{equation}

\begin{equation}
    \Delta t < \frac{1}{2 \pi \omega_{max}}
\end{equation}

\begin{equation}
    \omega = \sqrt{\frac{k_{axial}}{m_{min}}}
\end{equation}

\section{参考}
\begin{thebibliography}{99}  
\bibitem{ref1}Amanda Ghassaei, Erik D. Demaine, Neil Gershenfeld, Fast, Interactive Origami Simulation using GPU Computation, 2018.

\end{thebibliography}

\end{document}
